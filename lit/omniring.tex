\newpage
\section{Omniring}
\label{scn:omniring}

Omniring \cite{Lai2019} was proposed as a novel RingCT scheme which (i) did not require any trusted setup,
(ii) had a proof size logarithmic in the size of the ring, and (iii) allowed to share the same
ring between all source addressed in a transaction, thereby enabling significantly improved privacy.
Omniring provides key insights in the direction of generalising Bulletproofs \cite{Bunz2018} framework for RingCT applications.

\subsection{Main Idea}
\label{subscn:idea_omniring}

We start with a ring $\mathcal{R} = (P_1, P_2, \dots, P_n) \in \G^n$ of public keys of the form $P_i = g^{x_i}$ where $g \in \G$ is the group generator and $x_i \in \Z_q$ is the secret key, for all $i \in [n]$.
Suppose we own the public keys (addresses) at indices $(i_1, i_2, \dots, i_s)$ such that each $i_j \in [n]$ for all $j \in [s]$ and $s < n$.
This implies that we know the secret keys $\textbf{x} = (x_{i_1}, x_{i_2}, \dots, x_{i_s})$.
We would like to prove the knowledge of tuples $(i_j, x_{i_j})$ for $j \in [s]$.
For all $j\in [s]$, let us define unit vectors $\textbf{e}_{j} \in \{0,1\}^n$ such that it has $1$ only in position $i_j$.
Therefore,
\begin{align}
    1 &= g^{-x_j} \mathcal{R}^{\textbf{e}_{j}} \text{ for all } j \in [s]. \nonumber \\
    \intertext{Combining the above equations using powers of a scalar challenge $u \in \Z_q$, we get}
    \implies 1 &= g^{- \langle \vecnb{u}^s, \textbf{x} \rangle} \cdot \mathcal{R}^{\sum_{j=1}^s u^{j-1} \textbf{e}_{j}}.
    \label{eqn:maineq_omniring}
\end{align}
We will refer to equation (\ref{eqn:maineq_omniring}) as the \textit{main equality}.
By embedding the bases $(g \| \mathcal{R})$ and secrets $\textbf{a} = (\textbf{e}_{1}, \textbf{e}_2, \dots, \textbf{e}_s, x_{i_1}, x_{i_2}, \dots, x_{i_s})$ in the main equality in form of a Pedersen vector commitment, we can build an inner product relationship between the secrets and challenges to use the Bulletproofs framework in building a RingCT.
However, the soundness of Bulletproofs is guaranteed because the elements in the base vectors used in the Pedersen vector commitment are independent and uniformly random.
In this case, the prover might know the discrete log relationship between some elements in $\mathcal{R}$ since he owns multiple addresses.
To overcome this, Lai et al. \cite{Lai2019} proposed using base vector of the form:
\begin{equation*}
    \textbf{g}_w \coloneqq (g \| \mathcal{R})^{\circ w} \circ \textbf{p},
\end{equation*}
where $w \in \Z_q$ and $\textbf{p} \rgen \G^{n+1}$. 
A \textsf{PPT} adversary cannot find the discrete log relation between the elements of the newly constructed base $\textbf{g}_w$ (refer to Lemma \ref{thmNoDLBase}).
Note that $\textbf{g}_w^{\textbf{a}} = \textbf{g}_{w^{\prime}}^{\textbf{a}}$ for any $w, w^{\prime} \in \Z_q$ because of the main equality.
Therefore, using the new base vector for two different challenges $w, w^{\prime} \in \Z_q$, we can run a Bulletproofs-like protocol twice and extract the secret vectors.
This ensures that the Bulletproofs-like protocol preserves soundness.



