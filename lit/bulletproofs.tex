\section{Bulletproofs}
\label{sec:bulletproofs}

Bulletproofs \cite{Bunz2018} is the state-of-art range proof with logarithmic communication size.
A range proof is a zero-knowledge proof showing that a given number lies in a particular range without revealing the number itself.
In this section, we review the logarithmic range proof protocol presented in Bulletproofs paper.

We wish to design a proof system for the following relation which is equivalent to the range proof language
\begin{align}
    \mathcal{L}_{\textsf{BP}} &= \big\{ (\underbrace{g,h, V \in \mathbb{G}, \ n \in \N}_{\textsl{crs}}; \ \underbrace{v, \gamma \in \Z_q}_{\textsl{wit}}): \ \underbrace{V = g^v h^{\gamma} \wedge v \in [0, 2^n)}_{\textsl{stmt}} \big\}
    \label{eqn:lang_bp}
\end{align}

Let $\textbf{a}_L = (a_1, \dots, a_n) \in \{0,1\}^n$ be the vector containing the bits of $v$, $v = \langle \textbf{a}_L,\textbf{2}^n \rangle$. 
Recall that $\textbf{2}^n = (1, 2^1, 2^2, \dots, 2^{n-1}) \in \Z_q^n$.
Prover $\mathcal{P}$ convinces the verifier that $v \in [0,2^n -1]$ by proving that:
\begin{enumerate}
    \item It knows $\textbf{a}_L \in \Z_q^{n}, \ v, \gamma \in \Z_q$ such that $V=g^vh^{\gamma}$
    \item $\langle \textbf{a}_L,\textbf{2}^n \rangle =v
    \ \text{and } \  \textbf{a}_L \circ \textbf{a}_R = \textbf{0}^n \ \text{and } \ 
    \textbf{a}_R = \textbf{a}_L - \textbf{1}^n$
\end{enumerate}

To do so, we take a random linear combination (chosen by the verifier) of the constraints to use inner-product argument. Note that $\langle \textbf{b},\textbf{y}^n \rangle = 0, y \in \Z_q \implies \textbf{b} = \textbf{0}^n$. For randomly chosen $y, z \in \Z_q$, we can write: 
\begin{align}
    z^2 \cdot \langle \textbf{a}_L,\textbf{2}^n \rangle +
    z \cdot \langle \textbf{a}_L - \textbf{1}^n - \textbf{a}_R,\textbf{y}^n \rangle +
    \langle \textbf{a}_L,\textbf{a}_R \circ \textbf{y}^n \rangle =
    z^2 \cdot v
\end{align}
\begin{align}
    \implies \
    \big\langle
    \underbrace{\textbf{a}_L - z\cdot \textbf{1}^n}_{\textsl{left secret}}, \
    \underbrace{\textbf{y}^n \circ (\textbf{a}_R + z\cdot \textbf{1}^n) + z^2 \cdot \textbf{2}^n}_{\textsl{right secret}}
    \big\rangle
    =
    z^2 + \delta(y,z)
    \label{iprf1}
\end{align}
\vspace{-5mm}
\begin{align*}
    \delta(y, z) 
    =
    (z-z^2)\cdot \langle \textbf{1}^n,\textbf{y}^n \rangle - z^3\langle \textbf{1}^n,\textbf{2}^n \rangle
\end{align*}

Here, $\delta(y,z) \in \Z_q$ is a quantity that the verifier can easily calculate since $y,z \in \Z_q^{\star}$ are the challenges generated by the verifier himself. So now, if the prover could send to the verifier the two vectors in the inner product in (\ref{iprf1}), then the verifier could check (\ref{iprf1}) using the commitment $V$ to $v$. Thus the verifier would be convinced that $v \in [0, 2^n-1]$. But there's an issue in this approach. The verifier can easily extract $\textbf{a}_L$ from the first vector which the prover sent for calculation of inner product. 

To address this issue, by introducing two additional blinding terms $\textbf{s}_L, \textbf{s}_R \in \Z_q^{n}$ to blind these vectors. Thus, we define two linear vector polynomials\\
$l(X), r(X) \in \Z_q^n [X]$ and $t(X) \in \Z_q$ as follows:
% \vspace{-4mm}
\begin{align*}
    l(X) &= \textbf{a}_L - z\cdot \textbf{1}^n + \textbf{s}_L \cdot X\\
    r(X) &= \textbf{y}^n \circ (\textbf{a}_R + z\cdot \textbf{1}^n + \textbf{s}_R\cdot X) + z^2 \cdot \textbf{2}^n\\
    t(X) &= \langle l(X),r(X) \rangle = t_0 + t_1 \cdot X + t_2 \cdot X^2
    \intertext{where}
    t_0 &= \big\langle
    \textbf{a}_L - z\cdot \textbf{1}^n, \
    \textbf{y}^n \circ (\textbf{a}_R + z\cdot \textbf{1}^n) + z^2 \cdot \textbf{2}^n
    \big\rangle\\
    t_1 &= \big\langle
    \textbf{a}_L - z\cdot \textbf{1}^n, \textbf{y}^n \circ \textbf{s}_R  
    \big\rangle
    +
    \big\langle
    \textbf{s}_L, \left( \textbf{y}^n \circ (\textbf{a}_R + z\cdot \textbf{1}^n) + z^2 \cdot \textbf{2}^n \right)
    \big\rangle
    \\
    t_2 &= \big\langle
    \textbf{s}_L, \textbf{y}^n \circ \textbf{s}_R
    \big\rangle
\end{align*}

The constant term of $\langle l(X),r(X) \rangle$ are the required inner product vectors in (\ref{iprf1}). Now, the prover can publish  $l(x)$ and $r(x)$ for one $x \in \Z_q$. 
The constant term of $t(X)$, denoted $t_0$, is the result of the inner product. The prover needs to convince the verifier that this $t_0$ satisfies $t_0 = z^2\cdot v + \delta(y,z)$. 
To so do, the prover commits to the remaining coefficients of $t(X)$, $t_1, t_2 \in \Z_q$.

\begin{figure}[h!]
\caption{Inner Product Range Proof for $\L_{\textsf{BP}}$}
\label{fig:protocol_BP}
\end{figure}
\vspace{-12pt}
\begin{mdframed}
\begin{itemize}[itemsep=4pt]
    % Setup Phase
    \item[] \textsl{Setup}($\lambda, \L_{\textsf{BP}}$):
    \\[-5pt]\rule{\textwidth}{0.4pt}\\ 
    Generate following elements randomly from $\G$: $h \rgen \G, \ \textbf{g}, \textbf{h} \rgen \G^{n}$
    \\[2pt]
    Set: (\textsl{crs, stmt, wit}) as defined in the language $\L_{\textsf{BP}}(\G, q, \textbf{g}, \textbf{h}, h)$ in (\ref{eqn:lang_bp})
    \vspace{2pt}

    % Prover
    \item[] $\langle \mathcal{P}(\textsl{crs, stmt, wit}), \mathcal{V}(\textsl{crs, stmt}) \rangle$ :
    \\[-5pt]\rule{\textwidth}{0.4pt}

    % \item[] $\mathcal{V}$: $u,v \rgen \Z$, $\mathcal{V} \longrightarrow \P$: $u, v$

    % \item[] $\mathcal{V} \longrightarrow \P$: $u$
    
    % \item[] $\P, \ \V$:
    % \begin{enumerate}[itemsep=5pt]
    %     \item[(i)] $\hat{I} \coloneqq \ \textbf{I}^{- \vecnb{u}^s}$
        
    %     \item[(ii)] For $w \in \Z_q$, write
    %     \vspace{-4pt}
    %     \begin{equation}
    %       \textbf{g}_{w} \coloneqq \big[ \big((g \| g_t \| \textbf{C} \| \hat{I})^{\circ w} \circ \textbf{p}\big) \|\textbf{g}^{\prime} \big]
    %     \end{equation}
    % \end{enumerate}

    \vspace{-4pt}
    \item[] $\P$:\vspace{-4pt}
    \begin{enumerate}[itemsep=5pt]
        \item[(i)] Set $\textbf{a}_L \in \{0,1\}^n$ such that $\textbf{a}_L, \textbf{2}^n = v$
        \item[(ii)] Set $\textbf{a}_R = \textbf{a}_L - \textbf{1}^n \in \mathbb{Z}_q^n$
        \item[(iii)] $\alpha \leftarrow \mathbb{Z}_q$ 
        \item[(iv)] $A = h^{\alpha}\textbf{g}^{\textbf{a}_L}\textbf{h}^{\textbf{a}_R}$
        \item[(v)] $\textbf{s}_L, \textbf{s}_R \leftarrow \mathbb{Z}_q^n$
        \item[(vi)] $\rho \leftarrow \mathbb{Z}_q$
        \item[(vii)] $S = h^{\rho}\textbf{g}^{\textbf{s}_L}\textbf{h}^{\textbf{s}_R}$ 
        \item[(viii)] $\tau_1, \tau_2 \leftarrow \mathbb{Z}_q$
        \item[(ix)] $T_i = g^{t_i}h^{\tau_i}$, $i \in \{ 1,2\}$ 
    \end{enumerate}

    \item[] $\mathcal{P} \longrightarrow \V$: $A, S, T_1, T_2 \in \G$ 

    \item[] $\mathcal{V}$: $x,y,z \rgen \Z_q$, $\mathcal{V} \longrightarrow \P$: $x,y,z$

    % \item[] $\mathcal{V} \longrightarrow \P$: $w$

    \item[] $\P$:\vspace{-4pt}
    \begin{enumerate}[itemsep=5pt]
        \item[(i)] $\textbf{l} = l(x) = \textbf{a}_L - z\cdot \textbf{1}^n + \textbf{s}_L \cdot x \ \in \mathbb{Z}_q^n$
        \item[(ii)] $\textbf{r} = r(x) = \textbf{y}^n \circ (\textbf{a}_R + z\cdot \textbf{1}^n + \textbf{s}_R\cdot X) + z^2 \cdot \textbf{2}^n \ \in \mathbb{Z}_q^n$
        \item[(iii)] $\hat{t} = \langle \textbf{l}, \textbf{r}\rangle \in \mathbb{Z}_q$
        \item[(iv)] $\tau_x = \tau_2\cdot x^2 + \tau_1\cdot x + z^2\cdot \gamma \in \mathbb{Z}_q$
        \item[(v)] $\mu = \alpha + \rho \cdot x \in \mathbb{Z}_q$
    \end{enumerate}

    \item[] $\mathcal{P} \longrightarrow \V$: $ \textbf{l},\textbf{r} \in \Z_q^n, \ \tau_x, \mu, \hat{t} \in \Z_q$
    
    \item[] $\mathcal{V}$: 
    \begin{enumerate}[itemsep=5pt]
        \item[i] $h^{\prime}_i = h_i^{(y^{-i+1})} \in \mathbb{G}, \forall i$
        \item[(ii)] $P = A\cdot S^x \cdot \textbf{g}^{-z} \cdot
        (\textbf{h}^{\prime})^{z\cdot \textbf{y}^n + z^2 \cdot \textbf{2}^n} \in \mathbb{G}$
          
        \labelText{\item[(iii)]}{label:veq1_bp}
        $\hat{t} \stackrel{?}{=}
        \langle \textbf{l},\textbf{r}\rangle \in \mathbb{Z}_q$
        \hfill{{\small \textit{// $\hat{t}$ was computed correctly}}}

        \labelText{\item[(iv)]}{label:veq2_bp}
        $ g^{\hat{t}}
        h^{\tau_x} 
        \stackrel{?}{=}
        V^{z^2} 
        \cdot
        g^{\delta(y,z)} 
        \cdot 
        T_1^{x} 
        \cdot 
        T_2^{x^2}$
        \hfill{{\small \textit{// $\hat{t}$ satisfies $t_0 + t_1x + t_2x^2$}}} 

        \labelText{\item[(v)]}{label:veq3_bp}
        $P \stackrel{?}{=}
        h^{\mu} \cdot \textbf{g}^{\textbf{l}} \cdot (\textbf{h}^{\prime})^{\textbf{r}}$
        \hfill{{\small \textit{// Check if $\lvec = l(x)$ and $\rvec = r(x)$}}}
        
    \end{enumerate}
\end{itemize}
\end{mdframed}
\vspace{1cm}
Some key observations from the above protocol in Figure \ref{fig:protocol_BP} are:
\begin{enumerate}
    \item In this protocol, $\mathcal{P}$ sends $(\textbf{l}, \textbf{r})$, whose size is linear in $n$.
    \item The total communication cost in this protocol is $2n+4$ elements in $\mathbb{G}$ and $3$ elements in $\mathbb{Z}_q$.
    \item The verifier, to check if the received $\textbf{l}, \textbf{r}$ are actually $l(x), r(x)$ and also check if $t(x)=\langle \textbf{l}, \textbf{r}\rangle$, first, switches the generators of the commitment from $\textbf{h} \in \mathbb{G}^n$ to $\textbf{h}^{\prime} = \textbf{h}^{(\textbf{y}^{-n})}$.   
    \item Now, $A$ becomes a Pedersen vector commitment to $(\textbf{a}_L, \textbf{a}_R \circ \textbf{y}^n)$ w.r.t $(\textbf{g}, \textbf{h}^{\prime}, h)$. Similarly, $S$ is now a Pedersen vector commitment to $(\textbf{s}_L, \textbf{s}_R \circ \textbf{y}^n)$.
    \item If all three conditions marked by red rectangle result in a positive answer to the verifier, (s)he accepts and thus prover succeeds.
\end{enumerate}

\begin{theorem}[Range Proof]
    The range proof presented above has perfect completeness, perfect special honest verifier zero-knowledge, and computational witness extended emulation.
    \label{thm:rp}
\end{theorem}
\textit{Proof:} The range proof is a special case of the aggregated range proof with $m=1$. Refer Appendix C in \cite{Bunz2018} for the proof of the theorem \ref{thm:rp}.

In the above protocol, prover had to transfer $\textbf{l} \in \Z_q^n$ and $\textbf{r} \in \Z_q^n$ resulting in proof size proporional to $2n$. 
For a proof whose size is logarithmic in $n$, we can eliminate the transfer of $\textbf{l}$ and $\textbf{r}$ using the inner-product argument. 
We also observe that vectors $(\textbf{l}, \textbf{r})$ are not secret and hence a protocol that only provides soundness\footnote{\textit{Soundness} is the property of only being able to prove "true" things. \textit{Completeness} is the property of being able to prove all true things. \cite{sc13}} is sufficient. 

Concretely, observe that verifying first and third checks of the protocol in Figure \ref{fig:protocol_BP} is the same as verifying that the witness $(\textbf{l}, \textbf{r})$ satisfies the inner product relation (\ref{eq:ipa1}) on public input $(\textbf{g}, \textbf{h}^{\prime}, Ph^{-\mu}, \hat{t})$, where $P$ is a Pedersen vector commitment to vectors $\textbf{l}, \textbf{r} \in \mathbb{Z}_q^n$ whose inner product is $\hat{t}$.
\vspace{-2mm}
\begin{align*}
        P & \stackrel{?}{=}  h^{\mu} \cdot \textbf{g}^{\textbf{l}} \cdot (\textbf{h}^{\prime})^{\textbf{r}}
        \\
        \hat{t} & \stackrel{?}{=} \langle \textbf{l},\textbf{r}\rangle \in \mathbb{Z}_q
        \\
        \vspace{2mm}
        \{ (\textbf{g}, \textbf{h} \in \mathbb{G}^n, P \in \mathbb{G}, c \in & \mathbb{Z}_q ;\ 
        \textbf{a}, \textbf{b} \in \mathbb{Z}_q^{n}) :\
        P = \textbf{g}^{\textbf{a}}\textbf{h}^{\textbf{b}} \wedge c = \langle \textbf{a}, \textbf{b} \rangle \}
    \end{align*}
    
Thus, using the inner product argument, the total communication cost reduces down to $\textcolor{black}{2\lceil \text{log}_2(n) \rceil + 4}$ elements in $\mathbb{G}$ and $\textcolor{black}{5}$ elements in $\mathbb{Z}_q$.
