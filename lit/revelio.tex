\section{\textnormal{{\fontfamily{qag}\selectfont Revelio}} - A MimbleWimble Proof of Reserves Protocol}
\label{scn:revelio}

\R \cite{Dutta2019b} was the first proof of reserves protocol for MimbleWimble backed cryptocurrencies.
It uses non-interactive zero-knowledge (NIZK) proofs of knowledge for proving statements involving discrete logarithms which were originally introduced by \cite{Camenisch1997}.
% It presented a ring signature based approach to design proof of reserves proving statement of the form: an exchange owns $s$-out-of-$n$ outputs in a given anonymity set of size $n$.
% We briefly discuss Ring signature and Linkable Ring signature schemes used by \R and then present the idea of \Rw.

% \subsection{Ring Signatures}

% Ring Signatures were first introduced by Rivest et al. \cite{Rivest2001} in a paper titled `How to Leak a Secret'.
% Ring signatures make it possible to specify a set of possible signers without revealing which member actually produced the signature.
% They do not require any setup or coordination as against group signatures.



\subsection{Proving Statements About Discrete Logarithms}

Let $\G$ be a cyclic group of prime order $q$. Let $G, G^{\prime}, H$ be the generators of the group such that the discrete log relation between any two of them is assumed to be unknown.
We will use additive notation to be consistent with the notation used in \R paper.
\R uses three particular types of NIZK proofs of knowledge as defined below.
Let $\H \{0,1\}^{*} \rightarrow \Z_q $ be a cryptographic hash function modelled as a random oracle. 
The scalar pair $(\alpha, \beta) \in \Z_q^2$ is called as the representation of $X \in \G$ with respect to generators $G,H$ such that $X = \alpha G + \beta H$.
\begin{definition}
    A pair of scalars $(c,s) \in \Z_q^2$ is a NIZK proof of knowledge of the discrete log of an element $X \in \G$ with respect to a generator $G$ if they satisfy
    $$ c = \H(G,X,sG + cX). $$We will denote such a pair by $PoK\{\alpha \ | \ X = \alpha G \}$.
\end{definition}

\begin{definition}
    A triple of scalars $(c,s_1,s_2) \in \Z_q^3$ is a NIZK proof of knowledge and equality of the representations of $X,Y \in \G$ with respect to generator pairs $(G,H)$ and $(G^{\prime}, H)$ respectively if they satisfy
    $$ c = \H(S, s_1G+ s_2H+ cX,s_1G^{\prime} + s_2H + cX). $$where $S = (G\|G^{\prime}\|H\|X\|Y)$. We will denote such a triple by 
    \begin{equation*}
        PoK\{\underbrace{(\alpha, \beta)}_{\textsl{wit}} \ | \ \underbrace{X = \alpha G + \beta H \ \wedge \ Y = \alpha G^{\prime} + \beta H}_{\textsl{stmt}} \}.
    \end{equation*}
\end{definition}

\begin{definition}
    A 5-tuple of scalars $(c_1,c_2,s_1,s_2,s_3) \in \Z_q^5$ is a NIZK proof of either
    \begin{enumerate}
        \item[(i)] the knowledge and equality of the representations of $X,Y \in \G$ with respect to generator pairs $(G,H)$ and $(G^{\prime}, H)$ respectively OR
        \item[(ii)] knowledge of the discrete logarithm of the element $Y \in \G$ with respect to generator $G^{\prime}$, 
    \end{enumerate}  
    if they satisfy
    $$ c_1 + c_2 = \H(S, V_1, V_2, V_3). $$where $S = (G\|G^{\prime}\|H\|X\|Y)$ and 
    \begin{align*}
        V_1 &= s_1G+ s_2H+ c_1X, \\
        V_2 &= s_1G^{\prime}+ s_2H+ c_1X, \\
        V_3 &= s_3G^{\prime}+ c_2Y
    \end{align*}
    We will denote such a triple by 
    \begin{equation*}
        PoK\{\underbrace{(\alpha, \beta, \gamma)}_{\textsl{wit}} \ | \ \underbrace{(X = \alpha G + \beta H \ \wedge \ Y = \alpha G^{\prime} + \beta H) \vee (Y = \gamma G^{\prime})}_{\textsl{OR stmt}} \}.
    \end{equation*}
\end{definition}

\subsection{Main Idea of \textnormal{{\fontfamily{qag}\selectfont Revelio}}}

To generate a \R proof, an exchange chooses an anonymity set $\C_{\text{anon}} = (C_1, C_2, \dots, C_n)$ from the Grin blockchain.
Let the set of outputs owned by the exchange be $\C_{\text{own}} \subset \C_{\text{anon}}$. 
For $C_i \in C_{\text{own}}$, the exchange knows the blinding factor $k_i \in\Z_q$ such that $C_i = k_iG + a_iH$.
For each $C_i \in C_{\text{anon}}$, the exchange also publishes a tag $I_i$ defined as
\begin{equation}
    I_i = 
    \begin{cases}
        k_iG^{\prime} + v_iH & \text{if } C_i \in C_{\text{own}}\\
        y_iG^{\prime} & \text{if } C_i \notin C_{\text{own}}
    \end{cases}
\end{equation}
where $y_i = \H(k_{\text{exch}}, C_i)$ and $k_{\text{exch}}$ is a long-term secret key of the exchange.
The reason for such a definition of $y_i$ is that it needs to be a determinitic function of the chosen cover output $C_i$.
This is because we need to have consistent tag for a particular cover output $C_i$ appearing in $\C_{\text{anon}}$ over multiple \R proofs.
Additionally, the exchange publishes NIZK PoK $\sigma_i = (c_1^i,c_2^i, s_1^i, s_2^i, s_3^i)$ of the form 
\begin{equation*}
    PoK\{(\alpha, \beta, \gamma) \ | \ (C_i = \alpha G + \beta H \ \wedge \ I_i = \alpha G^{\prime} + \beta H) \vee (I_i = \gamma G^{\prime}) \}.
\end{equation*}
Lastly, the exchange claims that the commitment $C_{\text{assets}} = \sum_{i \in [n]} I_i$
is a Pedersen commitment to the total assets owned by the exchange.
Note that the tag list $(I_1, I_2, \dots, I_n)$ is used to check collusion among exchanges.
A \R proof verification involves checking of any matches in tags of different exchanges and verification of the NIZK PoKs $\sigma_i \ \forall i \in [n]$.

\subsection{Drawbacks of \textnormal{{\fontfamily{qag}\selectfont Revelio}}}

The size of a \R proof is $5n$ elements in $\Z_q$ and $n+1$ elements in $\G$.
Privacy provided by \R for exchange-owned outputs directly depends on the number of cover addresses used.
An exchange would ideally like to have the entire UTXO set as the anonymity set. 
Since the proof sizes in \R increase linearly
with the size of $C_{\text{anon}}$, setting $C_{\text{anon}}$ to be equal
to the whole UTXO set is not a scalable strategy.

The collusion-resistance property of \R works only
if all the exchanges generate their proofs using the same
blockchain state.
Cryptographically enforcing simultaneous proof generation is necessary to avoid cheating by exchanges.
Both of the above drawbacks led us to the development of \RB which solves the above drawbacks with a higher computational cost.
We describe \RB in the next chapter. 