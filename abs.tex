%============================= abs.tex================================
\begin{Abstract}
The rise of cryptocurrencies began with the inception of Bitcoin in 2009. Since
then, several cryptocurrencies with better privacy and security guarantees are being
developed. Cryptocurrencies gained popularity among general masses with the establishment of cryptocurrency exchanges.
Also known as digital currency exchanges or crypto exchanges, they are essentially businesses that allow customers to trade
cryptocurrencies or digital currencies for other assets including conventional fiat
money or different digital currencies. 
From a customer point of view, exchanges not only made owning cryptocurrencies possible to non-miners but also provided them with fast trading platforms for transactions within cryptocurrencies and fiat.
Customers were also provided with custodial wallets freeing them from the hassle of storing and remembering private keys.
In the early days of cryptocurrency, crypto exchanges were very few and less-known, but not too long ago their number increased dramatically and they became an integral part of the cryptoeconomic ecosystem. 
They were responsible for the boost in the transaction volumes of the vast majority of the cryptocurrency sales and liquidity.

The downside of such cryptocurrency exchanges is that they are required to
store sensitive information of customers like the private keys and account balances.
If in case an exchange is hacked, it might result in loss of customer-owned cryptocurrency assets.
There have been many high-profile hacks over the years, many of which went unnoticed for some time \cite{Cryptohacks}.
Although having a fool-proof method to avoid such hacks might be a difficult task, proof of reserves is one way to uphold the trust of customers.
A proof of reserves is the guarantee by the exchange that it owns reserves at least as much as its total liabilities towards customers. 
In this way, even after cases of hacking, the exchange could repay its liabilities to the customers.

The simplest way to publish a proof of reserves for an exchange is to reveal all the addresses or account details it owns so that the customers are convinced about the assets owned by the exchange.
Another way could be to send all the reserves it owns from all its addresses to a single addresses it owns.
If amounts involved in a transaction are public as in the case of Bitcoin, such a self-transaction would be a proof of the exchange's reserves.
For example, in 2011, Mt.~Gox cryptocurrency exchange transferred 424,242 bitcoins from its wallets to a previously revealed Bitcoin address \cite{MtGoxWikipedia}.
However, such proofs of reserves do not preserve the privacy of the exchanges. 
Information of an exchange's addresses or accounts and the total assets it owns are crucial for aspects of its business.
Exchanges naturally would not be in a position to compromise such critical information as a part of proofs of reserves.
The main challenge in design of proofs of reserves is to preserve privacy and confidentiality of exchanges but at the same time convince customers about an exchange's actual asset ownership.
Regaining \textit{trust} of the customers without compromising exchanges' \textit{privacy} is the primary motivation behind the design of better proofs of reserves. 
Advanced cryptographic techniques make it possible to design proofs of reserves which reveal \textit{nothing} beyond an assertation of the form:
\begin{center}
    \texttt{Exchange ABC owns ? amount of Y cryptocurrency and here is a proof $\Pi$ for customers to verify.}
\end{center}
Note that here we do not intend to reveal even the total amount. 
The proof $\Pi$ is a cryptographic tool known as a \textit{Non-Interactive Zero-Knowledge} proof.

In this work, we study the existing proof of reserves protocols for privacy-centric cryptocurrencies Grin, Beam and Monero.
The existing proof of reserves protocols possess some shortcomings which becomes a hurdle in their practical deployment.
With an aim to alleviate limitations of previously designed proofs of reserves, we design novel proofs of reserves for crypto exchanges supporting the above cryptocurrencies.
Our protocols are shorter and privacy enhancing in comparison to the existing state-of-the-art proofs of reserves.
We also implement the protocols we design as well as the previous state-of-the-art protocols for comparison and show feasibility in practical deployment of our protocols.
We believe that our work on proof of reserves could be well adopted by crypto exchanges and benefit several of their customers.
Furthermore, our work also opens up avenues of exploration of establishing trust without compromising privacy or anonymity in a more general setting.
We are hopeful that this work acts as a small but crucial contribution towards the goal of establishing a \textit{trustless, decentralized economy.}
%
%
%
%
%
\end{Abstract}
%=======================================================================

