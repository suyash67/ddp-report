%============================= abs.tex================================
\begin{Abstract}

The source of new cryptocurrencies is via the mining rewards for mining new blocks on the blockchain.
Cryptocurrency exchanges enable customers to buy digital assets without mining them in exchange for fiat currencies.
They also provide customers with custodial wallets for trading purposes. 
The convenience provided by such exchanges sometimes causes loss of customers assets in cases of hacks and frauds.
From the customer point of view, it is expected that the exchanges pay for the lost assets in unfortunate cases of exchanges being hacked.
Proof of Solvency is a technique to prove that an exchange owns assets at least as much as its liabilities.
If exchanges regularly publish proofs of solvency, they can regain the trust of their customers.
A proof of solvency comprises of a proof of reserves and a proof of liabilities.
Our work focuses only on designing privacy-preserving proof of reserves protocols for cryptocurrency exchanges as it depends only on publicly available data from the blockchain as against private customer data in the latter.

In this work, we study the existing proof of reserves protocols for privacy-centric cryptocurrencies Grin, Beam and Monero.
The existing proof of reserves protocols possess some shortcomings which becomes a hurdle in their practical deployment.
With an aim to alleviate limitations of previously designed proofs of reserves, we design novel proofs of reserves for MimbleWimble-based cryptocurrencies.
Our protocol is shorter and privacy enhancing in comparison to the existing state-of-the-art proof of reserves protocols.
Previous state-of-the-art proof of reserves protocol provided some privacy to exchanges by hiding the exchange-owned outputs in a larger anonymity set. 
The proof sizes for this protocol scaled linearly with the anonymity set size.
Since the level of privacy in a proof of reserves directly depends on how large the size of the anonymity set size is as compared to the number of exchange-owned outputs, larger anonymity sets imply stronger privacy.
However, linear dependence of the proof sizes on the anonymity set bought practical limitations (with regards to that of proof storage and broadcast) on the level of privacy that could be attained.  
With an aim to improve scalability of the previously design proof of reserves protocols, we design a strategy based on Bulletproofs technique \cite{Bunz2018} to design proofs of reserves scaling logarithmically in the anonymity set.
This brings flexibility in the choice of the size of anonymity set and therefore enhances the attainable level of privacy.
Along with this, we also use the concept of \textit{key images} to ensure that different exchanges cannot share their addresses.
The key images subsequently also helps in detecting double-spending of addresses by an exchange.
Going a step further, we also devise a cryptographic technique to enforce different exchanges to publish proofs of reserves corresponding to a same blockchain state, which is absent in previous work.
This ensures that exchanges can publish proofs of reserves only at particular time instances (for example, after each block is mined), preventing them from cheating customers with ambiguous choices of anonymity sets.   
We also implement the protocol we design and demonstrate feasibility in practical deployment of our protocol.
We believe that our work on proof of reserves could well be adopted by crypto exchanges and benefit several of their customers.
Our technique can be generalized for other privacy-focussed cryptocurrencies like Monero.
% We are currently working on extending this to Monero and give a brief description of the same in Appendix.
Furthermore, our work also opens up avenues of exploration of establishing trust without compromising privacy or anonymity in a more general setting.
% We are hopeful that this work acts as a small but crucial contribution towards the goal of establishing a \textit{trustless, decentralized economy.}

Pedersen commitments are perfectly hiding, i.e. no adversary, irrespective of the computational power he possesses, can find the amount hidden in a Pedersen commitment.
Amounts in Grin are hidden in outputs which are Pedersen commitments.
In this work, we also parallely explore if the transaction structure in Grin reveals any information about the amounts hidden in the outputs.

%
%
%
%
%
\end{Abstract}
%=======================================================================

